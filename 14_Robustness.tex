%!TEX root = optimization1718.tex

\chapter{Robustness}
\emph{Speakers: Amartya Sanyal, Mario Lezcano Casado and David Martínez Rubio }\\

% Change of notation with respect to the paper to maintain the symbols we have been using in the reading group:
%   Paper                          Us
% --------------------------------------------------------------------------------
%   R(\theta)                      r(\theta)         - True risk
%     -                            R(\theta)         - Empirical risk
%   R(\theta, \mathcal{P}_n)     R(\theta, \rho)   - robust risk
%
%


\section{Introduction}
The generalization problem we have mostly been focusing on. Let $\mathcal{X}$ be a sample space $P$ an unknown distribution on $\mathcal{X}$ from which we can sample and $\Theta$ a parameter space. Given a loss function $\ell : \Theta \times \mathcal{X} \rightarrow \mathcal{R}$, we want to minimize over $\Theta$ the risk
\[
    r(\theta) := \mathbb{E}_{X\sim P} [\ell(\theta, X)]\\
\]

for solving this task, we take samples $x_1, \dots, x_n$ i.i.d. from $P$ and then one usually compute the empirical risk
\[
    R(\theta) := \frac{1}{n} \sum_{i=1}^n \ell(\theta, x_i)
\]
and minimize it over $\Theta$. In the literature we can find in different bounds contexts bounds of the following form, with high probability:
\[
    r(\theta) \leq \frac{1}{n} \sum_{i=0}^n \ell(\theta, x_i) + C_1 \sqrt{\frac{\Var(\ell(\theta, X))}{n}} + \frac{C_2}{n} \text{ for all } \theta \in \Theta
\]
where $C_1$ and $C_2$ depend on the parameters of the problem and the desired confidence guarantee. These bounds justify empirical risk minimization (ERM), i.e. minimizing $R(\theta)$. ERM is the most natural way to tackle our problem, we take an unbiased estimator of the value we want to minimize and minimize that one instead. However, while choosing $\theta$ we do not make any attempt to control the variance of our losses and thus the quantity we really want to bound could be big. 

It seems natural to ask if we can get trade-offs between bias and variance in the generalization error, or more concretely if minimizing the weighted sum
\begin{equation}\label{empirical_biased_estimator}
R(\theta) + C\sqrt{\frac{\Var_{\hat{P}_n}(\ell(\theta, X))}{n}}
\end{equation}
we can get a lower generalization error. This idea was considered previously, but unfortunately the previous expression is not necessarily convex when $\ell$ is convex.  The paper we present, \textit{Variance-based regularization with convex objectives} \cite{duchi17roubust} proposes a different estimator, convex on $\theta$, that under some assumptions it is close to \eqref{empirical_biased_estimator}. They call their estimator the \textit{robustly regularized risk} and prove some generalization guarantees depending on the Rademacher complexities, covering numbers or localized Rademacher complexities.

\section{Robustly Regularized Risk}

Before defining the robustly regularized risk, we need to define first some concepts.

Let $\phi : \mathbb{R}_+ \rightarrow \mathbb{R}$ be a convex function with $\phi(1) = 0$. Then the $\phi$-\textit{divergence} between distributions $P$ and $Q$ defined on a space $\mathcal{X}$ is 
\[
    D_{\phi}(P\|Q) = \int \phi\left(\frac{dP}{dQ} dQ\right) = \int_{\mathcal{X}} \phi\left(\frac{p(X)}{q(X)}\right) q(X) d\mu(X),
\]
where $\mu$ is any measure for which $P,Q  \ll \mu$, and $p= \frac{dP}{d\mu}$, $q = \frac{dQ}{d\mu}$. The paper makes use of the $\chi^2$-divergence, which corresponds with $\phi(t)=(t-1)^2$. Given $\phi$ and a sample $X_1, \dots, X_n$, we define the \textit{local neighborhood of the empirical distribution with radius} $\rho$ by 
\[
    \mathcal{P}_n := \left\{ \text{distributions } P \text{ such that } D_\phi\left(P \| \hat{P}_n\right)\leq \frac{\rho}{n } \right\}
\]
where $\hat{P}_n$ denotes the empirical distribution of the sample. Then, the robustly regularized risk is defined as
\[
    R_n(\theta, \rho) := \sup_{P\in\mathcal{P}_n} \mathbb{E}_P [\ell(\theta, X)] = \sup_{P}\left\{\mathbb{E}_P[\ell(\theta, X)]: D_{\phi}(P\|\hat{P}_n) \leq \frac{\rho}{n}\right\}.
\]
It is convex on $\theta$ because it is defined as the supremum of convex functions. The proposed estimation is to choose a parameter $\hat{\theta}^\rob$ that minimizes $R_n(\theta, \rho)$. The authors prove that this estimator is close to \eqref{empirical_biased_estimator}, in particular it is $O(1/n)$ lower than  \eqref{empirical_biased_estimator} uniformly in $\theta$ and if the variance of $\ell(\theta, X)$ is big enough we will see that with high probability the two estimators are the same.

\section{Variance expansion}
We want to see that the risk $R_n(\theta, \rho)$ is a good approximation to the empirical risk plus a variance term. In order to understand this kind of variance expansion it is useful to study first the following problem
\begin{align}\label{variance_expansion}
	\begin{aligned}
         &\text{maximize}_{p} \sum_{i=1}^n p_i z_i \\
         &\text{subject to } p \in \mathcal{P}_n = \left\{p\in \mathbb{R}_+^n : \frac{1}{2} \|np-\textbf{1}\|_2^2 \leq \rho, \langle \textbf{1}, p \rangle = 1\right\},
	\end{aligned}
\end{align}
where $z \in \mathbb{R}^n$ is a vector. Let $s_n^2 := \frac{1}{n} \|z\|_2^2 - (\bar{z})^2 $ be the empirical variance of the vector $z$, where $\bar{z} = \frac{1}{n} \langle \textbf{1}, z\rangle$ is the mean value of $z$. Then by introducing the variable $u=p-\frac{1}{n} \textbf{1}$, the objective in problem \eqref{variance_expansion} satisfies $\langle p, z \rangle = \bar{z} + \langle u, z \rangle = \bar{z} + \langle u, z-\bar{z} \rangle$ because $\langle u, \textbf{1}=0\rangle$. Thus problem \eqref{variance_expansion} is equivalent to solving
\[
    \text{maximize}_{u\in \mathbb{R}^n}  \bar{z} + \langle  u, z-\bar{z}\rangle \text{  subject to } \|u\|_2^2 \leq \frac{2\rho}{n^2}, \langle \textbf{1}, u\rangle = 0, u \geq -\frac{1}{n}.
\]
If there is $u \propto z-\bar{z}$ that satisfies the constraints, the maximum is attained at such vector. This would be 
\[
    u_i = \frac{\sqrt{2\rho}(z_i -\bar{z})}{n\|z-\bar{z}\|_2} = \frac{\sqrt{2\rho} (z_i -\bar{z})}{n\sqrt{ns^2_n}}.
\]
It is possible to satisfy the constraint $u_i \geq -1/n$  if and only if 
\[
    \min_{i\in [n]} \frac{\sqrt{2\rho}(z_i - \bar{z})}{\sqrt{ns_n^2}} \geq -1.
\]
Thus if variance is large enough and thus the previous inequality holds for the vector $z$ we have 
\[
    \sup_{p\in\mathcal{P}_n} \langle p, z\rangle = \bar{z} + \sqrt{\frac{2\rho s_n^2}{n}}.
\]
This will correspond to an equality between $R_n(\theta, \rho)$ and \eqref{variance_expansion} when the variance is large enough.
